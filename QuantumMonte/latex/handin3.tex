\documentclass[twocolumn]{article}
\usepackage[english]{babel}
\usepackage[utf8]{inputenc}
\usepackage{amsmath,amssymb,physics,mathtools,blindtext,graphicx}
\usepackage[a4paper,total={7.5in,10in}]{geometry}
\usepackage[labelfont=bf]{caption}

\begin{document}
\begin{large}
\section*{Variational quantum Monte Carlo}
\subsection*{Introduction}
The variational quantum Monte Carlo method (VMC) is a powerful tool to find the ground states of quantum systems. It is especially applicable to many body systems for which analytical approaches, or even direct numerical integration techniques, are impossible. In this report, a despription will be given on the testing of the VMC method on two systems:
\begin{itemize}
    \item[1.] A one dimensional harmonic oscillator.
    \item[2.] A two particle, two dimensional system in which two electrons sit in a harmonic potential and interact through the Coulomb potential. 
\end{itemize}
The method was used in combination with a golden section search algorithm (GSS) in order to optimize the variational parameter of the trial functions used.

\subsection*{Method}
The energy of the ground state, $E_0$, is a lower bound for the local energy, $E_L$, of any trial function $\psi$:
\begin{equation}
    E_0\leq E_L = \frac{H\psi}{\psi}.
\end{equation}
In the VMC method, the positions $x$ are sampled according to the distribution $|\psi|^2$ by means of the Metropolis algorithm. The objective is to find the optimal trial function which gives a global minimum in the average of the local energy. The trial function can depend on a set of parameters which can be optimized so that the local energy is minimal. In the trial functions used in this report, a single variational parameter was used and a golden section search was used to optimize it. The general procedure looked like this:
%The local energy can either be calculated directly or approximated using some finite difference method. 
\begin{itemize}
    \item[1.] Give an interval $[a,b]$ for the variational parameter $\alpha$ where the minimum of $\langle E_L(\alpha)\rangle$ presumably lies. 
    \item[2.] Define two new points on the interval according to the golden section criteria, $c$ and $d$ for which $c<d$, and do Monte Carlo simulations for the trial function with $c$ and $d$ as variational parameters.
    \item[3.] If $\langle E_L(c)\rangle<\langle E_L(d)\rangle$, let $b=d$. Otherwise, let $a=c$. Then start again at 2 and continue until a given tolerance or number of iterations is reached. 
\end{itemize}
In the following results, the Monte Carlo steps were always around 100 000 were the first 10 000 steps were discarded. The trial displacement of the electrons was chosen so that the acceptance ratio was approximately 0.5.


\subsection*{Results for simple harmonic oscillator}
\begin{figure}[!t]
    % 80000
    \includegraphics[scale=0.35]{ParamsEnergyHarOsc.png}
    \caption{The average local energy and its variance in a quantum Monte Carlo simulation for the harmonic oscillator, as a function of the variational parameter $\alpha$. The points were generated with the golden section search algorithm.}
    \label{10apr1617}
    \includegraphics[scale=0.35]{DensityHarOsc.png}
    \caption{The distribution of the positions in a quantum Monte Carlo simulation for the harmonic oscillator. The sum of the histograms is normalized to 1. The variational parameter in the trial function was $\alpha=1/2$. The true probability density of the ground state is included.}
    \label{11apr0918}
\end{figure}
The hamiltonian of the first system studied is given by
\begin{equation}
    H = -\frac{1}{2}\frac{\text{d}^2}{\text{d}x^2} + \frac{x^2}{2}
\end{equation}
where reduced units are used in which length is measured in the units of $\sqrt{\hbar/m\omega}$ and energy in the units of $\hbar\omega$. The trial function is assumed to be a gaussian with a variational parameter $\alpha$:
\begin{equation}
    \psi(x,\alpha) = e^{-\alpha x^2}.
\end{equation}
The local energy of this trial function is then:
\begin{equation}
    E_L(x,\alpha) =\frac{H\psi}{\psi} = \left(1-4\alpha^2\right)x^2/2 + \alpha
\end{equation}
which takes the constant value $E_L=\hbar\omega/2$ for $\alpha = 1/2$, which corresponds to the true ground state wave function of the harmonic oscillator. 

The average of the local energy and its variance as functions of the variational parameter, obtained with the combination of VMC and GSS can be seen in figure \ref{10apr1617}. It can be inferred that the minimum average and variance of the local energy occurs when $\alpha\approx 1/2$. The distribution of the positions of the Monte Carlo simulation for $\alpha=1/2$ is plotted as a histogram in figure \ref{11apr0918}.
\subsection*{Results for harmonic Coulomb potential}
\begin{figure}[!b]
    % 200 000 iters
    \includegraphics[scale=0.35]{ParamsEnergylambda1.png}
    \caption{The average local energy and its variance as a function of the variational parameter $\alpha$ for coupling constant $\lambda=1$.}
    \label{11apr0919}
\end{figure}
The hamiltonian for the second system is given by
\begin{equation}
    H = -\frac{1}{2}\big(\boldsymbol{\nabla}_1^2+\boldsymbol{\nabla}_2^2\big) + V_\text{HO} + V_\text{C}.
    %\varphi(\mathbf{x}_1,\mathbf{x}_2)
\end{equation}
\begin{figure}[!b]
    \includegraphics[scale=0.35]{LambdaRelationship.png}
    \caption{The estimated ground state energies for different values of the coupling constant $\lambda$.}
    \label{11apr2039}
    \includegraphics[scale=0.35]{distances.png}
    \caption{The distributions of the square distances between the electrons for different coupling constants $\lambda$ as sampled by a Monte Carlo simulation.}
    \label{11apr2214}
    \includegraphics[scale=0.35]{2dhist.png}
    \caption{Two dimensional histograms of the position of an electron as sampled by the Monte Carlo method for different coupling constants $\lambda$. Darker regions indicate higher densities.}
    \label{11apr2215}
\end{figure}
The positions of the electrons are $\mathbf{x}_1 = (x_1,y_1)$ and $\mathbf{x}_2 = (x_2,y_2)$ and the potentials in the hamiltonian are defined by
\begin{equation}
    \begin{split}
        V_\text{HO} = \frac{m\omega^2}{2}\left(r_1^2+r_2^2\right), \quad\quad V_\text{C}  = \frac{\lambda}{r_{12}}
    \end{split}
\end{equation}
where,
\begin{equation}
    \begin{split}
        & \lambda = \frac{me^2\sqrt{\hbar/m\omega}}{4\pi\epsilon_0\epsilon_r\hbar^2}, \\ 
        &r_1^2 = x_1^2+y_1^2, \quad r_2^2 = x_2^2+y_2^2, \\ 
        &r_{12}^2 = (x_1-x_2)^2+(y_1-y_2)^2. 
    \end{split}
\end{equation}
The parameter $\lambda$ is dimesnionless and is a measure of the strength of the Coulomb interaction. 


The trial function used in the Monte Carlo simulation was:
\begin{equation}
    \psi(x,\alpha) = e^{-(r_1^2+r_2^2)/2}e^{\lambda r_{12}/(1+\alpha r_{12})}
\end{equation}
where $\alpha$ is the variational parameter. The local energy $E_L=H\psi/\psi$ was calculated using a second order finite difference formula for the second derivatives:
\begin{equation}
    \begin{split}
        \frac{\partial^2\psi}{\partial x_1^2}(x_1,y_1,x_2,y_2) &\approx \frac{1}{h^2}\big[\psi(x_1+h,x_2,y_1,y_2) \\ 
        &\hspace{1.3cm}-2\psi(x_1,y_1,x_2,y_2) \\ 
        &\hspace{1.3cm}+\psi(x_1-h,y_1,x_2,y_2) \big]
    \end{split}
\end{equation}
and likewise for the other partial derivaitves. The step size used was $h=10^{-3}$. Testing this trial function for $\lambda=0$ gave $\langle E_L\rangle = 2$ which is the ground state energy of a two praticle, two dimensional harmonic oscillator. 

The method was then tested for $\lambda\neq 0$. As in the Monte Carlo simulation of the one dimensional harmonic oscillator, the variational parameter $\alpha$ was varied with a golden section search in order to find the minimum of the local energy for a particular value of $\lambda$. The result for the case $\lambda = 1$ can be seen in figure \ref{11apr0919}. Similar plots were generated for other $\lambda$ and the minimum of local energy was estimated for each case. The results are summarized in figure \ref{11apr2039}. As $\lambda$ increases, the Coulomb repulsion grows and the particles are less likely to be found close to each other. This is demonstrated in figures \ref{11apr2214} and \ref{11apr2215}.

The combination of the VMC and the GSS worked rather well in most cases. A clear local minimum was obtained in all cases. The calculation of the local energy could have been more accurate by finding the closed expression of the second derivative of the trial function or by using a more accurate finite difference formula. However, since the second derivatives are only evaluated at points generated from the Monte Carlo simulation, the step size could be chosen to be very small. There was however no observable difference in choosing smaller step sizes than $h=10^{-3}$. 





\end{large}
\end{document}
