\documentclass[twocolumn]{article}
\usepackage[english]{babel}
\usepackage[utf8]{inputenc}
\usepackage{amsmath,amssymb,physics,mathtools,blindtext,graphicx,float}
\usepackage[a4paper,total={7.5in,10in}]{geometry}
\usepackage[labelfont=bf]{caption}

\begin{document}
\begin{large}
\begin{equation}
    \label{30apr1122}
    \frac{\text{d}^2(rV(r))}{\text{d}r^2} = -\frac{r\rho(r)}{\epsilon_0}
\end{equation}

\begin{equation}
    \label{30apr1123}
    V_{ee}^{exch}(r) = -\frac{3e}{4\pi\epsilon_0}\left(\frac{3\rho(r)}{8\pi e}\right)^{1/3}
\end{equation}

Each triplet $(n,\ell,m)$ can be occupied by zero, one or two electrons. The total charge distribution is then the sum of the contributions from each electron:
\begin{equation}
    \rho(r,\theta,\varphi) = \sum_{n=1}^{N}\sum_{\ell=0}^{n-1}\sum_{m=-\ell}^{\ell}g_{n\ell m}\rho_{n\ell m}(r,\theta,\varphi)
\end{equation}
where $g_{n\ell m}$ takes the value zero, one or two which is the number of electrons occupying the $(n,\ell,m)$ state, and 
\begin{equation}
    \begin{split}
        \rho_{n\ell m}(r,\theta,\varphi) &= -e\left|\psi_{n\ell m}(r,\theta,\varphi)\right|^2 \\ 
        &= -eR^2_{n\ell}(r)\left|Y_{\ell m}(\theta,\varphi)\right|^2. 
    \end{split}
\end{equation}
We'll replace the spherical harmonic with its average
\begin{equation}
    \begin{split}
        &\left|Y_{\ell m}(\theta,\varphi)\right|^2 \\ 
        &\hspace{0.3cm}\to \frac{1}{4\pi}\int\limits_{\theta=0}^\pi\int\limits_{\varphi=0}^{2\pi}\left|Y_{\ell m}(\theta,\varphi)\right|^2r^2\sin\theta\text{d}\theta\text{d}\varphi = \frac{1}{4\pi}
    \end{split}
\end{equation}
so that $\rho_{n\ell m} \to eR^2_{n\ell}/4\pi$. We notice that different $m$ give the same charge distribution for fixed $n$ and $\ell$, i.e. $\rho_{n\ell m} = \rho_{n\ell m'}$. Thus, we define $g_{n\ell}$ to be the number electrons being in a state with principal and azimuthal quantum number $(n,\ell)$ so that we can write 
\begin{equation}
    \label{30apr1108}
    \rho(r) = -\frac{e}{4\pi}\sum_{n=1}^{N}\sum_{\ell=0}^{n-1}g_{n\ell}R^2_{n\ell}(r)
\end{equation}
where $\rho(r)$ is to be understood as the approximation of $\rho(r,\theta,\varphi)$ in which we have averaged over the spherical harmonics.

The radial functions are solutions to 
\begin{equation}
    \label{30apr1107}
    \begin{split}
        &-\frac{\hbar^2}{2m}\frac{\text{d}^2\left(rR_{n\ell}\right)}{\text{d}r^2}  \\ 
        &\hspace{0.0cm} + \left[\frac{\hbar^2\ell(\ell+1)}{2mr^2}-\frac{Ze^2}{4\pi\epsilon_0r}-eV_{ee}(r)\right](rR_{n\ell}) = E(rR_{n\ell})  
    \end{split}
\end{equation}
where $P_{n\ell} = rR_{n\ell}$ is the reduced radial wavefunction and $V_{ee}(r)$ is a yet unknown electric potential. 

\begin{itemize}
    \item[1.] Set $V_{ee}=0$.
    \item[2.] Calculate $\rho(r)$ by solving \eqref{30apr1107} for the different $n$ and $\ell$ and calculating the sum in \eqref{30apr1108}.
    \item[3.] Use $\rho(r)$ to solve Poisson's equation \eqref{30apr1122} and obtain from it the potential $V_{ee}^{dir}$.
    \item[4.] Calculate the exchange potential $V_{ee}^{exch}$ using equation \eqref{30apr1123} and set $V_{ee} = V_{ee}^{dir} + V_{ee}^{exch}$. 
    \item[5.] Repeat from 2. until convergence of energy $E_{n\ell}$. 
\end{itemize}

\subsection*{Notes}
With $V_{ee}=0$, we get a numerical solution on $(0,1)$, $u(\xi)$, where $\xi = r/a_0D$. Then,
\begin{equation}
    rR_{n\ell}(r) = \frac{u_{n\ell}(r/a_0D)}{\sqrt{a_0D}}
\end{equation}
The distribution is then 
\begin{equation}
    \rho(r) = -e\sum_{n,\ell}g_{n\ell}R_{n\ell}^2.
\end{equation}
Insert this into 
\begin{equation}
    \begin{split}
        0 &= \frac{\text{d}(rV_{ee}^{dir})}{\text{d}r^2} + r\rho(r)/\epsilon_0  \\ 
        &= \frac{1}{(a_0D)^2}\frac{\text{d}^2(\alpha\lambda)}{\text{d}\xi^2} - \frac{e}{\epsilon_0}\sum_{n,\ell}g_{n\ell}\frac{u_{n\ell}^2(\xi)}{(a_0D)^2\xi}
    \end{split}
\end{equation}
so set $\alpha=-e/\epsilon_0$ such that
\begin{equation}
    \frac{\text{d}^2\lambda}{\text{d}\xi^2} + \sum_{n,\ell}g_{n\ell}\frac{u_{n\ell}^2(\xi)}{\xi} = 0
\end{equation}
Which becomes 
\begin{equation}
    \lambda'' + \xi\sigma(\xi) = 0
\end{equation}
where 
\begin{equation}
    \sigma(\xi) = \sum_{n,\ell} g_{n\ell}\frac{u^2_{n\ell}}{\xi^2}
\end{equation}
The boundary conditions are $\lambda(0) = 0$ and $\lambda(1) = N/4\pi$ (check). From this we have
\begin{equation}
    rV_{ee}^{dir} = \alpha\lambda(r/a_0D)
\end{equation}
or 
\begin{equation}
    V_{ee}(r) = -\frac{e}{\epsilon_0r}\lambda(r/a_0D)
\end{equation}
which should be negative.

The new radial wavefunction is then given by
\begin{equation}
    \begin{split}
        &-\beta^2\left[u'' - \frac{\ell(\ell+1)}{\xi^2}u + \frac{2}{\beta}\left(\frac{1}{\xi}-\frac{4\pi\lambda(\xi)}{Z\xi}\right)u\right] \\ 
        &\hspace{1cm}= E'u
    \end{split}
\end{equation}

Defining $u_{n\ell}$ to be the numerical solution to $rR_{n\ell}$ and making the substitutions
\begin{equation}
    \xi = r/a_0,\quad E' = E/(\hbar^2/a_0^2m),
\end{equation}
the radial function becomes
\begin{equation}
    -\frac{1}{2}u'' + \left[\frac{\ell(\ell+1)}{2\xi^2}-\frac{mea_0^2}{\hbar^2}(\varphi_C+\varphi_{ee})\right]u = E'u
\end{equation}
The electric potential $\varphi_{ee}$ is the solution to the Poisson equation:
\begin{equation}
    (r\varphi_{ee})'' + r\rho(r)/\epsilon_0 = 0
\end{equation}
Now, we define 
\begin{equation}
    \hat{\varphi}_{ee} = \frac{\varphi_{ee}}{\hbar^2/(ma_0^2e)}
\end{equation}
and set $\sigma = \rho/B$ where $B$ is to be determined. Making these substituions, and setting $r=a_0\xi$, we get
\begin{equation}
    (\xi\hat{\varphi}_{ee})'' + \frac{4\pi a_0^3}{e}B\sigma(\xi)\xi = 0
\end{equation}
So if we let $B=e/(4\pi a_0^3)$, we get 
\begin{equation}
    (\xi\hat{\varphi}_{ee})'' + \xi\sigma(\xi) = 0
\end{equation}
The Coulomb part is always the same:
\begin{equation}
    \hat{\varphi}_C = \frac{\varphi_C}{\hbar^2/(ma_0^2e)} = \frac{Z}{\xi}
\end{equation}




\end{large}
\end{document}
