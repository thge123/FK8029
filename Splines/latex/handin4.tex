\documentclass[twocolumn]{article}
\usepackage[english]{babel}
\usepackage[utf8]{inputenc}
\usepackage{amsmath,amssymb,physics,mathtools,blindtext,graphicx}
\usepackage[a4paper,total={7.5in,10in}]{geometry}
\usepackage[labelfont=bf]{caption}

\begin{document}
\begin{large}
\section*{Solving the spherically symmetric Poisson's equation using B splines}
\subsection*{Introduction}
Poisson's equation is given by
\begin{equation}
    \nabla^2\varphi = -\rho/\epsilon_0
\end{equation}
where $\varphi$ is the electric potential, $\rho$ is a charge distribution and $\epsilon_0$ is the permittivity of free space. For a spherically symmetric problem, the equation simplifies to
\begin{equation}
    \frac{\text{d}^2\varphi}{\text{d}r^2}+\frac{2}{r}\frac{\text{d}\varphi}{\text{d}r} = -\rho(r)/\epsilon_0
\end{equation}
and by defining the function $u=r\varphi$, this becomes
\begin{equation}
    \label{12apr1845}
    \frac{\text{d}^2u}{\text{d}r^2} + r\rho(r)/\epsilon_0 = 0.
\end{equation}
A description will now be given on how this equation was solved numerically using B-splines for the following charge distributions:
\begin{itemize}
    \item[1.] A uniformily charged sphere, 
    \begin{equation}
        \rho(r) = 
        \begin{cases}
            3q/(4\pi R^3),\quad r\leq R \\ 
            0,\hspace{1.7cm}\quad r>R
        \end{cases}
    \end{equation}
    where $q$ is the total charge and $R$ is the radius of the sphere.
    \item[2.] A uniformily charged shell,
    \begin{equation}
        \rho(r) = 
        \begin{cases}
            0,\hspace{3.2cm}\quad r<R_1  \\ 
            3q/(4\pi (R_2^3-R_1^3)),\quad R_1\leq r\leq R_2 \\ 
            0,\hspace{3.2cm}\quad r>R_2
        \end{cases}
    \end{equation}
    where $R_1$ and $R_2$ are the inner and outer radii, respectively.
    \item[3.] The electron charge distribution in an hydrogen atom:
    \begin{equation}
        \rho(r) = \frac{q}{\pi a_0^3}e^{-2r/a_0}
    \end{equation}
    where $q$ is the charge of an electron and $a_0$ is the Bohr radius.
\end{itemize}


\subsection*{Method}
Since $\varphi(r) = u(r)/r$ and we wish $\varphi(0)$ to be finite, we let $u(0) = 0$. For the first and second charge distributions, we know that $u(r) = q/4\pi\epsilon_0$ outside the enclosing volumes because of Gauss's law. Accordingly, we let $u(R) = q/4\pi\epsilon_0$ for the first distribution and $u(R_2) = q/4\pi\epsilon_0$ for the second distribution. For the third distribution, we introduce a cut-off so that $u(\alpha a_0) = q/4\pi\epsilon_0$ where $\alpha$ is some constant which will later be determined such that the error from this approximation becomes negliable. To facilitate the numerical caluclations, the following dimensionless variables were defined:
\begin{equation}
    \begin{split}
        &\xi = r/r_0 \\ 
        &\lambda = u/(q/4\pi\epsilon_0)
    \end{split}
\end{equation}
where $r_0$ is equal to $R$ for the first distribution, $R_2$ for the second distribution and $\alpha a_0$ for the third distribution. Then equation \eqref{12apr1845} can be written as
\begin{equation}
    \lambda''(\xi) + \xi\sigma(\xi) = 0
\end{equation}
where $\sigma = 4\pi r_0^3\rho/q$. The boundary conditions for $\lambda$ are $\lambda(0) = 0$ and $\lambda(1)=1$. Defining a new function, $g(\xi) = \lambda(\xi) - \xi$, we obtain the following boundary value problems:
\begin{equation}
    \begin{split}
        &g''(\xi) + \xi\sigma(\xi) = 0 \\ 
        &g(0) = g(1) = 0.
    \end{split}
\end{equation}
These are the equations that were solved for the different charge distributions, $\sigma$.

To do so, a collocation method with B-splines as candidate solutions was used. The numerical solution to $g$ was written as 
\begin{equation}
    \hat{g}(\xi) = \sum_{j=0}^{n-1}c_jB_{j,k}(\xi)
\end{equation}
where $B_{j,k}$ are B-splines of order $k=4$. Five knot points were placed at $\xi=0$ and another five at $\xi=1$ so that the only non-zero B-spline at $\xi=0$ was $B_{1,k}$ and the only non-zero B-spline at $\xi=1$ was $B_{n,k}$. The boundary conditions were thus satisfied by setting $c_1=c_{n-1}=0$. The collocation points were Chebyshev nodes, 
\begin{equation}
    \xi_k = \frac{1}{2} + \frac{1}{2}\cos\left(\frac{\pi(2k-1)}{2(n-2)}\right),
\end{equation}
where $k=1,2,\dots n-2$. The coefficients were then obtained by solving the linear system of equations:
\begin{equation}
    \begin{split}
        &c_0 = 0, \quad c_{n-1} = 0, \\ 
        &\sum_{j=0}^{n-1}c_jB''_{j,k}(\xi_k) + \xi_k\sigma(\xi_k) = 0,\quad k=1,2\dots,n-2. \\ 
    \end{split}
\end{equation}

\subsection*{Results}














\end{large}
\end{document}
